%%%%%%%%%%%%%%%%%%%%%%%%%%%%%%%%%%%%%%%%%%%%%%%%%%%%%%%%%%%%%%%%%%%%%%%%%%%%%%%%
%2345678901234567890123456789012345678901234567890123456789012345678901234567890
%        1         2         3         4         5         6         7         8

\documentclass[letterpaper, 10 pt, conference]{ieeeconf}  % Comment this line out
                                                          % if you need a4paper
%\documentclass[a4paper, 10pt, conference]{ieeeconf}      % Use this line for a4
                                                          % paper

\IEEEoverridecommandlockouts                              % This command is only
                                                          % needed if you want to
                                                          % use the \thanks command
\overrideIEEEmargins
% See the \addtolength command later in the file to balance the column lengths
% on the last page of the document

% This is needed to prevent the style file preventing citations from linking to 
% the bibliography
\makeatletter
\let\NAT@parse\undefined
\makeatother

\usepackage[dvipsnames]{xcolor}

\newcommand*\linkcolours{ForestGreen}

\usepackage{times}
\usepackage{graphicx}
\usepackage{amssymb}
\usepackage{gensymb}
\usepackage{amsmath}
\usepackage{breakurl}
\def\UrlBreaks{\do\/\do-}
\usepackage{url,hyperref}
\hypersetup{
colorlinks,
linkcolor=\linkcolours,
citecolor=\linkcolours,
filecolor=\linkcolours,
urlcolor=\linkcolours}

\usepackage{algorithm}
\usepackage{algorithmic}

\usepackage[labelfont={bf},font=small]{caption}
\usepackage[none]{hyphenat}

\usepackage{mathtools, cuted}

\usepackage[noadjust, nobreak]{cite}
\def\citepunct{,\,} % Style file defaults to listing references separately

\usepackage{tabularx}
\usepackage{amsmath}

\usepackage{float}

\usepackage{pifont}% http://ctan.org/pkg/pifont
\newcommand{\cmark}{\ding{51}}%
\newcommand{\xmark}{\ding{55}}%

\newcommand*\diff{\mathop{}\!\mathrm{d}}
\newcommand*\Diff[1]{\mathop{}\!\mathrm{d^#1}}
\newcommand*\imgres{600}

\newcommand*\GitHubLoc{https://github.com/linted/allocation}

\newcolumntype{Y}{>{\centering\arraybackslash}X}

%\usepackage{parskip}

\usepackage[]{placeins}

% \usepackage{epstopdf}
% \epstopdfDeclareGraphicsRule{.tif}{png}{.png}{convert #1 \OutputFile}
% \AppendGraphicsExtensions{.tif}

\newcommand\extraspace{3pt}

\usepackage{placeins}

\usepackage{tikz}
\newcommand*\circled[1]{\tikz[baseline=(char.base)]{
            \node[shape=circle,draw,inner sep=0.8pt] (char) {#1};}}
            
\usepackage[framemethod=tikz]{mdframed}

\usepackage{afterpage}

\usepackage{stfloats}

\usepackage{atbegshi}
\newcommand{\handlethispage}{}
\newcommand{\discardpagesfromhere}{\let\handlethispage\AtBeginShipoutDiscard}
\newcommand{\keeppagesfromhere}{\let\handlethispage\relax}
\AtBeginShipout{\handlethispage}

\usepackage{comment}

\newcommand*\todo[0]{\textcolor{red}{TODO }}

%\usepackage[1,2,3,5,6,7]{pagesel} %Discard page 4 as it is blank

% The following packages can be found on http:\\www.ctan.org
%\usepackage{graphics} % for pdf, bitmapped graphics files
%\usepackage{epsfig} % for postscript graphics files
%\usepackage{mathptmx} % assumes new font selection scheme installed
%\usepackage{times} % assumes new font selection scheme installed
%\usepackage{amsmath} % assumes amsmath package installed
%\usepackage{amssymb}  % assumes amsmath package installed

\title{\LARGE \bf
Stack and Heap Allocations: A Cost Comparison
}

%\author{ \parbox{3 in}{\centering Huibert Kwakernaak*
%         \thanks{*Use the $\backslash$thanks command to put information here}\\
%         Faculty of Electrical Engineering, Mathematics and Computer Science\\
%         University of Twente\\
%         7500 AE Enschede, The Netherlands\\
%         {\tt\small h.kwakernaak@autsubmit.com}}
%         \hspace*{ 0.5 in}
%         \parbox{3 in}{ \centering Pradeep Misra**
%         \thanks{**The footnote marks may be inserted manually}\\
%        Department of Electrical Engineering \\
%         Wright State University\\
%         Dayton, OH 45435, USA\\
%         {\tt\small pmisra@cs.wright.edu}}
%}

\author{Michael D. Merrill$^{1}$% <-this % stops a space
\thanks{$^{1}$Michael is a Masters student in Computer Sciences, Georgia Tech,
and a developer for the 782nd MI BN's CSD-G.
Email: michael.d.merrill28.civ@mail.mil}%
}


\begin{document}


\maketitle
\thispagestyle{empty}
\pagestyle{empty}


%%%%%%%%%%%%%%%%%%%%%%%%%%%%%%%%%%%%%%%%%%%%%%%%%%%%%%%%%%%%%%%%%%%%%%%%%%%%%%%%
\begin{abstract}
Stack and Heap allocations have different allocation costs and use cases.
Incorrect choice of allocation method is often problematic and cause increases in latency,
loss of data, and program instability. I compare the choice of allocation method across various 
implementations and situations to show the cost benefit trade offs of each. 
Finally a quantitative explanation of the results through a high level description of the allocation algorithm internals is provided.
The source code is publicly available at \url{\GitHubLoc}.

\end{abstract}

%%%%%%%%%%%%%%%%%%%%%%%%%%%%%%%%%%%%%%%%%%%%%%%%%%%%%%%%%%%%%%%%%%%%%%%%%%%%%%%%
\section{INTRODUCTION}

Allocation algorithms are designed to limit the overhead related to memory management.
Namely, to be computationally inexpensive, effective for any allocation size, and to provide discrete blocks of memory. This paper addresses the different costs incurred through the use of the stack and heap across their various allocation methods. Costs measured include performance, flexibility, and code complexity.

A number of test algorithms will be used as measurements. Each will attempt to measure a discrete feature and function common to all allocation algorithms. Care is taken to ensure that tests do not favor one method over another through intentional use of compiler hints and directives. Each test algorithm is designed to mimic common, real world use cases of allocated memory. 

GCC is used to compile the test algorithms \todo.

% \begin{figure}[tbp]
% \centering
% \includegraphics[width=0.97\columnwidth]{loss_spikes.png}
% \caption{ Learning curve with high loss spikes that excessively perturb a trainable parameter distribution. Losses decrease after loss spikes as parameters are updated back to an intelligent distribution. The learning curve is 2500 iteration boxcar averaged. }
% \label{loss_spikes}
% \end{figure}


%%%%%%%%%%%%%%%%%%%%%%%%%%%%%%%%%%%%%%%%%%%%%%%%%%%%%%%%%%%%%%%%%%%%%%%%%%%%%%%%
\section{Algorithms}

Allocation with Light Usage Algorithm captures the use case of a 

% Adaptive learning rate clipping (ALRC, algorithm~\ref{alrc_algorithm}) is designed to addresses the limitations of gradient clipping. Namely, to be computationally inexpensive, effective for any batch size, robust to hyperparameter choices and to preserve backpropagated gradient distributions. Like gradient clipping, it also has to be applicable to arbitrary loss funtions and neural network architectures.  

\begin{algorithm}[h]
\caption{Allocation with Light Usage}
\begin{algorithmic}
\STATE let $T$ be the total number of successful iterations
\STATE let $0 < S$ be the number of bytes allocated
\STATE let $0 \leq R \leq S$ a random index to test
\STATE let $0 < C$
\WHILE{Test is not finished}
  \STATE $M \leftarrow allocate(S)$
  \IF{$M_\text{R} \neq C $}
  \STATE $M_\text{R} \leftarrow C$
  \STATE $C \leftarrow C + 1$
%   \ELSE\STATE $C$
  \ENDIF
  \STATE $deallocate(M)$
\ENDWHILE
\end{algorithmic}
\label{light_usage_algorithm}
\end{algorithm}

% Rather than allowing loss spikes to destabilize learning, ALRC applies the mapping $\eta L \rightarrow \text{stop\_gradient}(L_\text{max}/L) \eta L$ if $L > L_\text{max}$. The function $\text{stop\_gradient}$ leaves its operand unchanged in the forward pass and blocks gradients in the backwards pass. ALRC adapts the learning rate to limit the effective loss being backpropagated to $L_\text{max}$. The value of $L_\text{max}$ is non-trivial for ALRC to complement existing learning algorithms. In addition to training stability and robustness to hyperparameter choices, $L_\text{max}$ needs to adapt to losses and learning rates as they vary. 

% In our implementation, $L_\text{max}$ is a number of standard deviations of the loss above its mean and requires five hyperparameters. There are two decay rates, $\beta_1$ and $\beta_2$, for exponential moving averages used to estimate the mean and standard deviation of the loss and a number, $n$, of standard deviations. Similar to batch normalization\cite{ioffe2015batch}, any decay rate close to 1 is effective e.g. $\beta_1 = \beta_2 = 0.999$. Performance does vary slightly with $n$; however, we find that any $n \approx 3$ is effective. Initial values for the running means, $\mu_1$ and $\mu_2$, where $\mu_1^2 < \mu_2$ also have to be provided. However, any sensible initial estimates larger than their true values are fine as $\mu_1$ and $\mu_2$ will decay to their correct values.

% ALRC can be extended to any loss function or batch size. For batch sizes above 1, we apply ALRC to individual losses, while $\mu_1$ and $\mu_2$ are updated with mean losses. ARLC can also be applied to loss summands; such as per pixel errors between generated and reference images, while $\mu_1$ and $\mu_2$ are updated with the mean errors.

%%%%%%%%%%%%%%%%%%%%%%%%%%%%%%%%%%%%%%%%%%%%%%%%%%%%%%%%%%%%%%%%%%%%%%%%%%%%%%%%
\section{Experiments: CIFAR-10 Supersampling}

% To invistagate the ability of ALRC to stabilize learning and its robustness to hyperperameter choices, we performed a series of toy experiments with networks trained to upsample CIFAR-10\cite{krizhevsky2014cifar, krizhevsky2009learning} images to 32$\times$32$\times$3 after downsampling to 16$\times$16$\times$3.

% \vspace{\extraspace}
% \noindent\textbf{Data pipeline:} In order, images were randomly flipped left or right, had their brightness altered, had their contrast altered, were linearly transformed to have zero mean and unit variance and bilinearly downsampled to 16$\times$16$\times$3.

% \begin{figure}[tbh!]
% \centering
% \includegraphics[width=0.57\columnwidth]{alrc_architecture.png}
% \caption{ Convolutional image 2$\times$ supersampling network with three skip-2 residual blocks. }
% \label{alrc_architecture}
% \end{figure}

% \vspace{\extraspace}
% \noindent\textbf{Architecture:} Images were upsampled and passed through the convolutional network in fig.~\ref{alrc_architecture}. Each convolutional layer is followed by ReLU\cite{nair2010rectified} activation, except the last. 

% \vspace{\extraspace}
% \noindent\textbf{Initialization:} All weights were Xavier\cite{glorot2010understanding} initialized. Biases were zero initialized.

% \vspace{\extraspace}
% \noindent\textbf{Learning policy:} ADAM optimization was used with the hyperparameters recommended in \cite{kingma2014adam} and a base learning rate of 1/1280 for 100000 iterations. The learning rate was constant in batch size 1, 4, 16 experiments and decreased to 1/12800 after 54687 iterations in batch size 64 experiments. Networks were trained to minimize mean squared or quartic errors between restored and ground truth images. ALRC was applied to limit the magnitudes of losses to either 2, 3, 4 or $\infty$ standard deviations above their running means. For batch sizes above 1, ALRC was applied to each loss individually.

% \begin{figure*}[tbh!]
% \vspace{0.7cm}
% {\centering
% \includegraphics[width=\textwidth]{alrc.png}
% \caption{ Unclipped learning curves for 2$\times$ CIFAR-10 upsampling with batch sizes 1, 4, 16 and 64 with and without adaptive learning rate clipping of losses to 3 standard deviations above their running means. Training is more stable for squared errors than quartic errors. Learning curves are 500 iteration boxcar averaged. }
% \label{fig:alrc}}
% \footnotesize
% \vspace{\baselineskip}
% Squared Errors\\
% \begin{tabular*}{\textwidth}{@{\extracolsep{\fill}}c|cccccccc}
% \hline
% \multicolumn{1}{c|}{}       & \multicolumn{2}{c}{Batch Size 1} & \multicolumn{2}{c}{Batch Size 4} & \multicolumn{2}{c}{Batch Size 16} & \multicolumn{2}{c}{Batch Size 64} \\
% Threshold & Mean         & \multicolumn{1}{c}{Std Dev}      & Mean       & Std Dev      & Mean       & \multicolumn{1}{c}{Std Dev}       & Mean      & Std Dev      \\ \hline
% 2 & 5.55 & 0.048 & 4.96 & 0.016 & 4.58 & 0.010 & - & - \\
% 3 & 5.52 & 0.054 & 4.96 & 0.029 & 4.58 & 0.004 & 3.90 & 0.013 \\
% 4 & 5.56 & 0.048 & 4.97 & 0.017 & 4.58 & 0.007 & 3.89 & 0.016 \\
% $\infty$ & 5.55 & 0.041 & 4.98 & 0.017 & 4.59 & 0.006 & 3.89 & 0.014 \\
% \hline
% \end{tabular*}
% \vspace{\baselineskip}

% Quartic Errors\\
% \begin{tabular*}{\textwidth}{@{\extracolsep{\fill}}c|cccccccc}
% \hline
% \multicolumn{1}{c|}{}       & \multicolumn{2}{c}{Batch Size 1} & \multicolumn{2}{c}{Batch Size 4} & \multicolumn{2}{c}{Batch Size 16} & \multicolumn{2}{c}{Batch Size 64} \\
% Threshold & Mean         & \multicolumn{1}{c}{Std Dev}      & Mean       & Std Dev      & Mean       & \multicolumn{1}{c}{Std Dev}       & Mean      & Std Dev      \\ \hline
% 2 & 3.54 & 0.084 & 3.02 & 0.023 & 2.60 & 0.012 & 1.65 & 0.011 \\
% 3 & 3.59 & 0.055 & 3.08 & 0.024 & 2.61 & 0.014 & 1.58 & 0.016 \\
% 4 & 3.61 & 0.054 & 3.13 & 0.023 & 2.64 & 0.016 & 1.57 & 0.016 \\
% $\infty$ & 3.88 & 0.108 & 3.32 & 0.037 & 2.74 & 0.020 & 1.61 & 0.008 \\
% \hline
% \end{tabular*}
% \captionof{table}{ Adaptive learning rate clipping (ALRC) for losses 2, 3, 4 and $\infty$ running standard deviations above their running means for batch sizes 1, 4, 16 and 64. ARLC was not applied for clipping at $\infty$. Each squared and quartic error mean and standard deviation is for the means of the final 5000 training errors of 10 experiments. ALRC lowers errors for unstable quartic error training at low batch sizes and otherwise has little effect. Means and standard deviations are multiplied by 100. }
% \label{table:alrc}
% % \vspace{0.3cm}
% \end{figure*}

% \vspace{\extraspace}
% \noindent\textbf{Results:} Example learning curves for mean squared and quartic error training are shown in fig.~\ref{fig:alrc}. Training is more stable and converges to lower losses for larger batch sizes. Training is less stable for quartic errors than squared errors, allowing ALRC to be examined for loss functions with different stability.

% Training was repeated 10 times for each combination of ALRC threshold and batch size. Means and standard deviations of the means of the last 5000 training losses for each experiment are tabulated in table~\ref{table:alrc}. ALRC has no effect on mean squared error (MSE) training, even for batch size 1. However, it decreases errors for batch sizes 1, 4 and 16 for mean quartic error training.

% \FloatBarrier
% \discardpagesfromhere
% \clearpage
% \keeppagesfromhere

% \begin{figure*}[tbh]
% \centering
% \includegraphics[width=0.93\textwidth]{alrc_example.png}
% \caption{ Neural network completions of 512$\times$512 scanning transmission electron microscopy images from 1/20 coverage blurred spiral scans. }
% \label{alrc_example}
% \end{figure*}

% \section{Experiments: Partial-STEM}

% To test ALRC in practice, we applied our algorithm to neural networks learning to complete 512$\times$512 scanning transmission electron microscopy (STEM) images from partial scans with 1/20 coverage. Example completions are shown in fig.~\ref{alrc_example}.

% \begin{figure*}[tbp!]
% \vspace{1.5cm}
% \centering
% \includegraphics[width=0.97\textwidth]{gen-2-step.png}
% \caption{ Two-stage generator that completes 512$\times$512 micrographs from partial scans. A dashed line indicates that the same image is input to the inner and outer generator. Large scale features developed by the inner generator are locally enhanced by the outer generator and turned into images. An auxiliary inner generator trainer restores images from inner generator features to provide direct feedback. }
% \label{gen-2-step}
% \vspace{1.5cm}
% \end{figure*}


% \vspace{\extraspace}
% \noindent\textbf{Data pipeline:} In order, each image was subject to a random combination of flips and 90$\degree$ rotations to augment the dataset by a factor of 8. Next, each STEM images was blurred and a path described by a 1/20 coverage spiral was selected. Finally, artificial noise was added to scans to make them more difficult to complete.

% \vspace{\extraspace}
% \noindent\textbf{Architecture:} Our network can be divided into the three subnetworks shown in fig.~\ref{gen-2-step}: an inner generator, outer generator and an auxiliary inner generator trainer. The auxiliary trainer\cite{szegedy2014going, szegedy2015rethinking} is introduced to provide a more direct path for gradients to backpropagate to the inner generator. Each convolutional layer is followed by ReLU activation, except the last.

% \vspace{\extraspace}
% \noindent\textbf{Initialization:} Weights were initialized from a normal distribution with mean 0.00 and standard deviation 0.05. There are no biases.

% \vspace{\extraspace}
% \noindent\textbf{Weight normalization:} All generator weights are weight normalized\cite{salimans2016weight} and a weight normalization initialization pass was performed after weight initialization. Following \cite{salimans2016weight, hoffer2018norm}, running mean-only batch normalization was applied to the output channels of every convolutional layer except the last. Channel means were tracked by exponential moving averages with decay rates of 0.99. Similar to \cite{chen2017rethinking}, running mean-only batch normalization was frozen in the second half of training to improve stability.

% \vspace{\extraspace}
% \noindent\textbf{Loss functions:} The auxiliary inner generator trainer learns to generate half-size completions that minimize MSEs from half-size blurred ground truth STEM images. Meanwhile, the outer generator learns to produce full-size completions that minimize MSEs from blurred STEM images. All MSEs were multipled by 200. The inner generator cooperates with the auxiliary inner generator trainer and outer generator. 

% To benchmark ALRC, we investigated training with MSEs, Huberized ($h=1$) MSEs, MSEs with ALRC and Huberized ($h=1$) MSEs with ALRC before Huberization. Training with both ALRC and Hubarization showcases the ability of ALRC to complement another loss function modification.

% \vspace{\extraspace}
% \noindent\textbf{Learning policy:} ADAM optimization\cite{kingma2014adam} was used with a constant generator learning rate of 0.0003 and a first moment of the momentum decay rate, $\beta_1=0.9$, for 250000 iterations. In the next 250000 iterations, the learning rate and $\beta_1$ were linearly decayed in eight steps to zero and 0.5, respectively. The learning rate for the auxiliary inner generator trainer was two times the generator learning rate; $\beta_1$ were the same. All training was performed with batch size 1 due to the large model size needed to complete 512$\times$512 scans.

% \begin{figure}[htbp]
% \centering
% \includegraphics[width=0.97\columnwidth]{stability.png}
% \caption{ Outer generator losses show that ALRC and Huberization stabilize learning. ALRC lowers final mean squared error (MSE) and Huberized MSE losses and accelerates convergence. Learning curves are 2500 iteration boxcar averaged. }
% \label{stability}
% \end{figure}

% \vspace{\extraspace}
% \noindent\textbf{Results:} Outer generator losses in fig.~\ref{stability} show that ALRC and Huberization stabilize learning. Further, ALRC accelerates MSE and Huberized MSE convergence to lower losses. To be clear, learning policy was optimized for MSE training so direct loss comparison is uncharitable to ALRC.

%%%%%%%%%%%%%%%%%%%%%%%%%%%%%%%%%%%%%%%%%%%%%%%%%%%%%%%%%%%%%%%%%%%%%%%%%%%%%%%%
\section{Discussion}

% Taken together, our CIFAR-10 supersampling results show that ALRC improves stability and lowers losses for learning that would be destabilized by loss spikes and otherwise has little effect. Loss spikes are often encountered when training with high learning rates, high order loss functions or small batch sizes. However, a moderate learning rate was used in MSE experiments so losses did not spike enough to destabilize learning. In contrast, mean quartic error training is unstable so ALRC stabilizes training and lowers losses. Similar results are confirmed for partial-STEM where ALRC stabilizes learning and lowers losses.

% ALRC is designed to complement existing learning algorithms with new functionality. It is effective for any loss function or batch size and can be applied to any neural network trained with a variant of stochastic gradient descent. Our algorithm is also computationally inexpensive, requiring orders of magnitude fewer operations than other layers typically used in neural networks. As ALRC either stabilizes learning or has little effect, this means that it is suitable for routine application to arbitrary neural network training with SGD. In addition, we note that ALRC is a simple algorithm that has a clear effect on learning.

% Nevertheless, ALRC can replace other learning algorithms in some situations. For instance, ALRC is a computationally inexpensive alternative to gradient clipping in high batch size training where gradient clipping is being used to limit perturbations by loss spikes. However, it is not a direct replacement as ALRC preserves the distribution of backpropagated gradients whereas gradient clipping reduces large gradients. Instead, ALRC is designed to complement gradient clipping by limiting perturbations by large losses while gradient clipping modifies gradient distributions.

% The implementation of ALRC in algorithm~\ref{alrc_algorithm} is for positive losses. This avoids the need to introduce small constants to prevent divide-by-zero errors. Nevertheless, ALRC can support negative losses by using standard methods to prevent divide by zero errors. Alternatively, a constant can be added to losses to make them positive without affecting learning.

% ALRC can also be extended to limit losses more than a number of standard deviations below their mean. This had no effect in our experiments. However, preemptively reducing loss spikes by clipping rewards between user-provided upper and lower bounds can improve reinforcement learning\cite{mnih2015human}. Subsequently, we suggest that clipping losses below their means did not improve learning because losses mainly spiked above their means; not below. Some partial-STEM losses did spike below; however, they were mainly for blank or otherwise trivial completions.

%%%%%%%%%%%%%%%%%%%%%%%%%%%%%%%%%%%%%%%%%%%%%%%%%%%%%%%%%%%%%%%%%%%%%%%%%%%%%%%%
\section{Conclusions}

% We have developed ALRC to stabilize the training of artificial neural networks by limiting backpropagated losses. Our experiments show that ALRC accelerates convergence and lowers losses for learning that would be destabilized by loss spikes and otherwise has little effect. Further, ALRC is computationally inexpensive, can be applied to any loss function or batch size, does not affect the distribution of backpropagated gradients and has a clear effect on learning. Overall, ALRC complements existing learning algorithms and can be routinely applied to arbitrary neural network training with SGD.

\section{Source Code}

Source code for all experiments detailed in this paper can be found at \url{\GitHubLoc}.

% ALRC increases the rate of convergence and decreases losses for otherwise unstable training by limiting loss spikes. If learning is already stable, it has little affect. Our algorithm is computationally inexpensive, robust to hyperparameter choices and can be applied to any loss function or batch size without affecting backpropagated gradient distributions. It follows that ALRC complements existing learning algorithms and can be applied to any artificial neural network trained with SGD.

%%%%%%%%%%%%%%%%%%%%%%%%%%%%%%%%%%%%%%%%%%%%%%%%%%%%%%%%%%%%%%%%%%%%%%%%%%%%%%%%

\bibliographystyle{ieeetr}
\bibliography{bibliography}

\section{Acknowledgements}

\noindent This research was funded by 782nd MI BN.

\clearpage

\end{document}
